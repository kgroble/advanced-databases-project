% Created 2017-03-31 Fri 20:35
% Intended LaTeX compiler: pdflatex
\documentclass[11pt]{article}
\usepackage[utf8]{inputenc}
\usepackage[T1]{fontenc}
\usepackage{graphicx}
\usepackage{grffile}
\usepackage{longtable}
\usepackage{wrapfig}
\usepackage{rotating}
\usepackage[normalem]{ulem}
\usepackage{amsmath}
\usepackage{textcomp}
\usepackage{amssymb}
\usepackage{capt-of}
\usepackage{hyperref}
\author{Coleman Gibson, Derek Kuhnert, Kieran Groble}
\date{\today}
\title{Project Proposal}
\hypersetup{
 pdfauthor={Coleman Gibson, Derek Kuhnert, Kieran Groble},
 pdftitle={Project Proposal},
 pdfkeywords={},
 pdfsubject={},
 pdfcreator={Emacs 26.0.50 (Org mode 9.0.4)},
 pdflang={English}}
\begin{document}

\maketitle

\section{Features}
\label{sec:orgd4a8a8d}
\subsection{Searching Users}
\label{sec:orgd67df88}
A main feature of our application will be to allow users to search for other
accounts based on multiple properties, including name, interests (e.g. Vim or
Emacs), and location. This will potentially result in a sort of user browser,
which users can use to find potential matches.

\subsection{Rank User Matches}
\label{sec:orgd87d086}
Something which could be used in numerous parts of our application is the
ability to rank user matches. This could be done off a number of factors,
including user interests and their preferences in what they are looking for
in other users, such as giving operating system preferences a high weight.

\subsection{Basic Messaging}
\label{sec:org3d252ce}
To make matching with other users useful, users of our app must be able to
contact their matches in some way. We plan to do this with a basic messaging
system.

\subsection{User Caching}
\label{sec:org565c37d}
To make our site faster, we hope to implement a simple form of caching for
user profiles. Users are likely to search for others far more than they will
update their own profile, so we plan on storing a preconstructed
representation of our users in a database with excellent read speeds. This
representation will then be invalidated on any updates and reconstructed on
request.


\section{Tools}
\label{sec:org17d197a}
\subsection{Front End}
\label{sec:orgd1539c8}
\subsubsection{React}
\label{sec:org07000eb}
All three of us in the group are interested in learning React.js. We will
use this with JavaScript, HTML, and CSS to make a basic user interface to
our database project. It would be fairly unreasonable to make a command line
dating application, no matter how nerdy our users are.

\subsection{Back End}
\label{sec:org280a44a}
\subsubsection{Python}
\label{sec:org8e29c84}
Python is a language which we know all of our databases support well and is
a language all of our group members are familiar with. This makes it a good
choice for our primary back end language.

\subsubsection{Flask}
\label{sec:orgf19f034}
We believe that a larger framework like Django would be overkill for a three
week project, but would not like to write our web server with no support. We
think that Flask will give a good middle ground for what will be a
relatively small project.

\section{Databases}
\label{sec:org9f33ca5}
\subsection{ArangoDB}
\label{sec:org69a850b}
This will be our main store for any data involving relationships. For
example, we will use this to store things like user operating system
preferences and matches between users. The graph structure this database
gives us will make it extremely easy to do things like match user preferences
and check the number of shared matches two users have.

\subsection{Redis}
\label{sec:orgbde3df8}
We will use Redis as a basic store for noncritical data like user sessions in
our site. We would also like to experiment with using Redis as a user cache
so that we can serve documents constructed from ArangoDB and MongoDB without
reconstructing them from the databases.

\subsection{MongoDB}
\label{sec:org4a5a135}
MongoDB will be used for storing user profiles (including name, age,
description) as well as the chat data for our basic messenger. In short, it
will be used to store data which does not have any relationships. This will
give us some of the write and read speed benefits of Mongo where we can use
them. We wont lose the ability to store relationships entirely as we will
still be using ArangoDB for our relational data.

\section{Visual Representation}
\label{sec:orga769d6e}

\subsection{ArangoDB}
\label{sec:orged9e0b8}
\begin{figure}[htbp]
\centering
\includegraphics[width=.9\linewidth]{./arango.png}
\caption{ArangoDB data model}
\end{figure}

\subsection{Redis}
\label{sec:org8a316e0}

\begin{verbatim}
<username>-session = user-session
<username>-cached-document = user-document
\end{verbatim}

\subsection{MongoDB}
\label{sec:org303ae3a}
\begin{verbatim}
user {
    username,
    name,
    birthday,
    description
}

chat {
    user1,
    user2,
    text
}
\end{verbatim}
\end{document}